\begin{frame}[t]
   \frametitle{X.509}
   \begin{itemize}
      \item issued in 1988
      \item binds public key to name
      \item hierarchical approach
      \item you generate your keys
      \begin{itemize}
         \item http://www.root.cz/clanky/jak-na-openssl-2
      \end{itemize}
      \item trusted CAs signs your keys
      \begin{itemize}
         \item after your verification
         \item after payment
      \end{itemize}
   \end{itemize}
\end{frame}

\begin{frame}[t]
   \frametitle{SSL}
   \textcolor{BlueViolet}{Client:} Hello

   \textcolor{DarkGreen}{Server:} Hello, my certificate is this give me yours

   \textcolor{BlueViolet}{Client:} Here is mine

   \textcolor{DarkSlateGray}{\textit{Client generates random number}}

   \textcolor{BlueViolet}{Client:} Let's agree on secret

   \textcolor{BlueViolet}{Client:} Here it is encrypted with your public key

   \textcolor{DarkSlateGrey}{\textit{Client and Server generates symmetric keys from secret}}

   \textcolor{BlueViolet}{Client:}\textcolor{red}{[cr]} So far we talked about \dots \textcolor{red}{[/cr]}

   \textcolor{DarkSlateGrey}{\textit{Server verifies}}
   
   \textcolor{DarkGreen}{Server}\textcolor{red}{[cr]}: So far we talked about \dots \textcolor{red}{[/cr]}
   
   \textcolor{DarkSlateGrey}{\textit{Client: verifies}}

   \textit{\dots data \dots}
\end{frame}
