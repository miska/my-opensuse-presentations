\sectionpage{RPM}

\begin{frame}[t]
   \frametitle{Basic information}
   \begin{itemize}
	\item one of the oldest in Linux world
	\item created in 1997 for RedHat
	\item used by various distributions
	\begin{itemize}
		\item RedHat/Fedora and derivates
		\item openSUSE/SLE, Mandriva/Mageia, Meego, \dots
	\end{itemize}
	\item several frontends to make operations easier
	\begin{itemize}
		\item yum - used by RedHat/Fedora and derivates
		\item zypper - used by openSUSE/SLE and Meego
		\item urpmi - used by Mandriva/Mageia
	\end{itemize}
   \end{itemize}
\end{frame}


\begin{frame}[t]
   \frametitle{File format}
   \begin{itemize}
      \item 32 bytes lead (magic, rpm version, type of package, architecture)
      \item signature
      \item header
      \begin{itemize}
         \item name, version and release
         \item license
         \item summary
         \item description
         \item changelog
         \item requires and provides
         \item file list with rights, md5s and more
      \end{itemize}
      \item archive
      \begin{itemize}
	\item cpio, compressed by gzip, bzip2, xz, \dots
      \end{itemize}
   \end{itemize}
\end{frame}


\begin{frame}[t]
   \frametitle{\texttt{.spec} file}
   \begin{itemize}
      \item source recipe for \texttt{rpm}
      \item contains multiple sections
      \begin{itemize}
         \item information about package
         \item \texttt{\%prep} section
         \item \texttt{\%build} section
         \item \texttt{\%install} section
         \item \texttt{\%check} section
         \item \texttt{\%files} section
         \item optionally more: \texttt{\%pre}, \texttt{\%post}, \dots
      \end{itemize}
   \end{itemize}
\end{frame}

\begin{frame}[fragile,t]
	\frametitle{Example .spec file (1/3)}
	\subsection{Example .spec file}
	\begin{small}
	\begin{verbatim}
Name:           nano  
BuildRequires:  ncurses-devel  
Url:            http://www.nano-editor.org/  
License:        GPL v3 or later  
Group:          Productivity/Editors/Other  
Summary:        Pico Editor Clone with Enhancements  
Version:        2.1.9  
Release:        1  
Source:         %{name}-%{version}.tar.bz2  
BuildRoot:      %{_tmppath}/%{name}-%{version}-build  

%description  
GNU nano is a small and friendly text editor. It aims to emulate the  
Pico text editor while also offering a few enhancements.
...

	\end{verbatim}
	\end{small}
\end{frame}
\begin{frame}[fragile,t]
	\frametitle{Example .spec file (2/3)}
	\begin{small}
	\begin{verbatim}
%prep  
%setup -q 
 
%build  
%configure --disable-rpath --enable-all  
%{__make} %{?jobs:-j%jobs}  
  
%install  
%make_install  
%find_lang %{name}
  
%check  
make check  

	\end{verbatim}
	\end{small}
\end{frame}
\begin{frame}[fragile,t]
	\frametitle{Example .spec file (2/3)}
	\begin{small}
	\begin{verbatim}
%files -f %{name}.lang  
%defattr(-, root, root)  
%doc AUTHORS BUGS COPYING ChangeLog INSTALL NEWS README THANKS TODO UPGRADE  
%{_mandir}/fr  
%{_mandir}/fr/man?  
%{_mandir}/man?/*.*  
%{_mandir}/*/man?/*.*  
%{_bindir}/*  
%{_infodir}/*.gz  
%{_datadir}/nano  
  
%changelog
	\end{verbatim}
	\end{small}
\end{frame}

