\documentclass{beamer}
\usetheme[pageofpages=of,% String used between the current page and the
                         % total page count.
          bullet=circle,% Use circles instead of squares for bullets.
          titleline=true,% Show a line below the frame title.
	  titlepagelogo=opensuse,
          alternativetitlepage=true,% Use the fancy title page.
          ]{Torino}
\usepackage[utf8]{inputenc}
\setbeamerfont{title}{series=\bfseries,size=\LARGE}
\author{Michal Hru\v{s}eck\'{y}\newline {\small openSUSE Boosters}}
\title{Jak na vlastní mrak}

\begin{document}

\begin{frame}[t,plain]
\titlepage
\end{frame}

\sectionpage{Trocha historie}

\begin{frame}
\frametitle{Cca pravěk}
\begin{center}
\includegraphics[height=.7\paperheight]{SeriousTux-RockArt-AncientAstronauts}
\end{center}
\end{frame}

\begin{frame}
\frametitle{Mobilní verze}
\begin{center}
\includegraphics[height=.7\paperheight]{j4p4n-egyption-tablet}
\end{center}
\end{frame}

\begin{frame}
\frametitle{Cca středověk}
\begin{center}
\includegraphics[height=.7\paperheight]{Anonymous_old_book}
\end{center}
\end{frame}

\begin{frame}
\frametitle{Mobilní verze}
\begin{center}
\includegraphics[height=.7\paperheight]{Degri_Notepad_Pencil}
\end{center}
\end{frame}

\begin{frame}
\frametitle{Minulé století}
\begin{center}
\includegraphics[height=.7\paperheight]{AJ_Computer}
\end{center}
\end{frame}

\begin{frame}
\frametitle{Mobilní verze}
\begin{center}
\includegraphics[height=.7\paperheight]{Palm}
\end{center}
\end{frame}

\begin{frame}
\frametitle{Extra mobilní verze}
\begin{center}
\includegraphics[height=.7\paperheight]{Cloud_computing}
\end{center}
\end{frame}

\sectionpage{ownCloud}

\begin{frame}[t]
\frametitle{Vlastnosti dat v běžných cloudech}
\begin{itemize}
\item[+] data dostupná odkudkoliv
\item[+] stejná data všude
\item[+] často možná synchronizace s offline úložištěm
\item[+] snadné sdílení
\item[--] cloudy bez synchronizace hrozí lock-inem
\item[--] správce cloudu o vás ví hodně
\item[--] bezpečnost vašich dat nezávisí na vás
\end{itemize}
\end{frame}

\begin{frame}[t]
\frametitle{Co je ownCloud}
\begin{itemize}
\item open source software
\item web aplikace
\item synchronizační klienti
\item každý může mít vlastní mrak
\item podporuje standardy
\begin{itemize}
\item soubory - WebDAV
\item kalendář - CalDAV
\item kontakty - CardDAV
\end{itemize}
\end{itemize}
\end{frame}

\sectionpage{Demo}

\begin{frame}[t]
\frametitle{Požadavky}
\begin{itemize}
\item webserver
\item php
\item SQL databáze (SQLite, MySQL, \dots)
\vspace{.5cm}
\item[\(\Rightarrow\)] stačí libovolný webhosting
\end{itemize}
\end{frame}


\sectionpage{Otazky?}

\end{document}

