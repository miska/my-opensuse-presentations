\documentclass{beamer}
\usetheme[pageofpages=of,% String used between the current page and the
                         % total page count.
          bullet=circle,% Use circles instead of squares for bullets.
          titleline=true,% Show a line below the frame title.
	  titlepagelogo=opensuse,
          alternativetitlepage=true,% Use the fancy title page.
          ]{Torino}

\usepackage{ulem}

\setbeamerfont{title}{series=\bfseries,size=\LARGE}
\author{Michal Hru\v{s}eck\'{y}}
\title{Recording TV}
\institute{openSUSE Boosters team}

\definecolor{darkgreen}{RGB}{0,99,0}
\definecolor{darkblue}{RGB}{0,0,99}
\definecolor{darkred}{RGB}{99,0,0}

\begin{document}

\titlepage

\sectionpage{Existing solutions}

\begin{frame}[t]
	\frametitle{MythTV}
	\begin{itemize}
		\item[+] Can be controlled from remote machine
		\item[*] Client-Server solution
		\item[--] Server requires MySQL
		\item[--] Configuration not really easy
		\item[--] Ugly default theme
	\end{itemize}
\end{frame}

\begin{frame}[t]
	\frametitle{MythTV}
	\begin{center}
	\includegraphics[width=.8\paperwidth]{mythtv}
	\end{center}
\end{frame}

\begin{frame}[t]
	\frametitle{XBMC}
	\begin{itemize}
		\item[+] Supports plugins
		\begin{itemize}
			\item[*] Python
			\item[*] Shell
		\end{itemize}
		\item[+] WebUI
		\item[+] Nice graphics
		\item[*] Standalone application
		\item[--] Doesn't support TV by itself
		\begin{itemize}
			\item[+] Plugins with hacks exists
		\end{itemize}
	\end{itemize}
\end{frame}

\begin{frame}[t]
	\frametitle{XBMC}
	\begin{center}
	\includegraphics[width=.8\paperwidth]{xbmc}
	\end{center}
\end{frame}

\begin{frame}[t]
	\frametitle{Kaffeine}
	\begin{itemize}
		\item[+] can work well in windowed mode
		\item[--] KDE program (lot of dependencies)
		\item[--] optimised for desktop
		\item[--] no remote control
		\item[--] AFAIK no easy way of remote control
	\end{itemize}
	\vfill
%	\hfill{\small\textcolor{blue}{\flushright\href{http://www.linuxexpres.cz/praxe/digitalni-televize-v-linuxu-a-prehravac-kaffeine}{www.linuxexpres.cz/praxe/digitalni-televize-v-linuxu-a-prehravac-kaffeine}}}
\end{frame}

%\begin{frame}[t]
%	\frametitle{Kaffeine}
%	\begin{center}
%	\includegraphics[height=.6\paperheight]{kaffeine}
%	\end{center}
%\end{frame}

\begin{frame}[t]
	\frametitle{Summary}
	\begin{itemize}
		\item There are many solutions available
		\item Only few were mentioned
		\item Nothing is perfect
	\end{itemize}

	\vspace{0.5cm}
	\textbf{My requirements:}
	\begin{itemize}
		\item Simple \& fast
		\item Easy remote control
		\item Recording multiple channels at once
		\item \textcolor{gray}{\textit{Easy to use}}
	\end{itemize}
\end{frame}

\sectionpage{Useful tools}

\begin{frame}[t,fragile]
	\frametitle{\texttt{mplayer}}
	\begin{itemize}
		\item plays movies/TV/everything
		\item runs from command line
		\item many options can be specified on command line
		\item support slave mode (controlled via fifo)
		\item comes with mencoder (can encode video files)
	\end{itemize}
	\vfill

	\hfill \textcolor{blue}{\href{ftp://ftp.mplayerhq.hu/MPlayer/DOCS/tech/slave.txt}{ftp://ftp.mplayerhq.hu/MPlayer/DOCS/tech/slave.txt}}
\end{frame}

\begin{frame}[fragile]
	\frametitle{\texttt{mplayer} - examples}
	\begin{verbatim}
mkfifo /tmp/slave.fifo
echo \
  'mplayer -slave -input file=/tmp/slave.fifo movie.mkv' \
    > ~/.xinitrc
startx


echo pause > /tmp/slave.fifo
echo seek +10 > /tmp/slave.fifo
	\end{verbatim}
\end{frame}

\begin{frame}[t,fragile]
	\frametitle{\texttt{at} - runs command at specified time}
	\begin{verbatim}
$ at 18:10
warning: commands will be executed using /bin/sh
at> mencoder -o tv.avi -oac copy -ovc copy dvb://CT1
at> <EOT>
job 4 at Sat Mar  5 18:10:00 2011

$ at 18:50
warning: commands will be executed using /bin/sh
at> killall mencoder
at> sleep 10
at> s2ram
at> <EOT>
job 5 at Sat Mar  5 18:50:00 2011
	\end{verbatim}
\end{frame}

\begin{frame}[fragile]
	\frametitle{\texttt{rtcwake} - sleeps for a while}
	\begin{verbatim}
# sleeps for 2,5 hours
rtcwake -m mem -s `expr 2 \* 3600 + 30 \* 60`



# will wake up at 18:00
rtcwake -m mem -t `date +%s -d '18:00'`
	\end{verbatim}
\end{frame}

\begin{frame}[t]
	\frametitle{\texttt{dvbstreamer}}
	Streams TV:
	\begin{itemize}
		\item Can get data from TV card
		\item Splits channels from multiplex
		\item Stream it over network
	\end{itemize}
	\vspace{0.5cm}
	Features:
	\begin{itemize}
		\item Client-Server application
		\item Can transmit multiple streams
		\item Can decode multiple channels from one multiplex
	\end{itemize}
\end{frame}

\begin{frame}[t,fragile]
	\frametitle{\texttt{dvbstreamer} - examples}
	\begin{verbatim}
dvb_ctrl() {
   echo "$*" | dvbctrl -a 0 -u user -p pass -f -
}

dvb_ctrl select 20cb.0111.0102
dvb_ctrl setmrl udp://127.0.0.1:8002
dvb_ctrl addsf CT1 udp://127.0.0.1:8001
dvb_ctrl setsf CT1 20cb.0111.0101
	\end{verbatim}
	\vfill

	\hfill\href{http://github.com/miska/dvb-scripts}{\textcolor{blue}{http://github.com/miska/dvb-scripts}}
\end{frame}

\begin{frame}[fragile]
	\frametitle{\texttt{netcat} - captures stream from network}
	\begin{verbatim}
# record data from dvbstreamer
netcat -l -u 127.0.0.1 -p 8001 | pv > ct1.avi


# transfer files
# send
cat ct1.avi | netcat -l 3333
# receive
netcat 192.168.1.80 3333 > backup.iso
	\end{verbatim}
\end{frame}

\begin{frame}[fragile]
	\frametitle{\texttt{ethtool} \& \texttt{wol}}
	\begin{verbatim}
# enable wake on lan on eth0
ethtool -s eth0 wol g


# wakeup video
wol -p 32767 -i 192.168.1.255 00:50:FC:FA:9A:FF
	\end{verbatim}
\end{frame}

\sectionpage{Questions?}

\end{document}

