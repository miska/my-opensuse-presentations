\documentclass{beamer}
\usetheme[pageofpages=of,% String used between the current page and the
                         % total page count.
          bullet=circle,% Use circles instead of squares for bullets.
          titleline=true,% Show a line below the frame title.
	  titlepagelogo=opensuse,
          alternativetitlepage=true,% Use the fancy title page.
          ]{Torino}

\setbeamerfont{title}{series=\bfseries,size=\LARGE}
\author{Michal Hru\v{s}eck\'{y}\newline {\small openSUSE Boosters}}
\title{Appstream}

\begin{document}

\begin{frame}[t,plain]
\titlepage
\end{frame}

\sectionpage{Package Management}

\begin{frame}[fragile]
\frametitle{Evolution - manual compilation}
\begin{center}
\begin{verbatim}
gcc -c model.c
gcc -c view.c
gcc -c controler.c
gcc *.o -o my_program
cp my_program /usr/local/bin
\end{verbatim}
\end{center}
\end{frame}

\begin{frame}[fragile]
\frametitle{Evolution - less manual compilation}
\begin{center}
\begin{verbatim}
./configure
make
make install
\end{verbatim}
\end{center}
\end{frame}

\begin{frame}[fragile]
\frametitle{Evolution - real package management}
\begin{center}
\begin{verbatim}
zypper in my_program
apt-get my_program
\end{verbatim}
\end{center}
\end{frame}

\begin{frame}
\frametitle{Evolution - user friendly package management}
\begin{center}
\includegraphics[height=6cm]{yast}

\small YaST screenshot by fsse8info
\end{center}
\end{frame}

\begin{frame}
\frametitle{Evolution - what users want?}
\begin{center}
\includegraphics[height=6cm]{steam}

\small Steam screenshot by factoryjoe
\end{center}
\end{frame}

\begin{frame}
\frametitle{Evolution - what users want?}
\begin{center}
\includegraphics[width=10cm]{play_store}
\end{center}
\end{frame}

\begin{frame}[t]
\frametitle{What are we missing?}
\begin{itemize}
\item many pictures, colors, advertisements, \dots
\pause
\vspace{0.5cm}
\item recommendations and suggestions
\item screenshots
\item rankings
\item reviews
\item users don't care about packages but about applications
\end{itemize}
\end{frame}

\sectionpage{Appstream}

\begin{frame}[t]
\frametitle{Appstream}
\begin{itemize}
\item cross distribution effort
\item focused on creating great upstream application installer
\item started with meeting in January 2011 in Nurnberg
\item people from Debian, Fedora, Mageia, openSUSE and Ubuntu
\item attempt to decide what, how and where to do and how to collaborate
\item trying to fix everything we were missing
\end{itemize}
\end{frame}


\begin{frame}[t]
\frametitle{General results}
\begin{itemize}
\item freedesktop.org and cross-distro mailing list are platforms to use
\item some metadata should be shared
\item working together makes sense
\end{itemize}

\vspace{1cm}

{\Large Technologies to use}
\begin{itemize}
\item Ubuntu Software Center
\item Debian screenshots application
\item Open Collaboration Services
\end{itemize}
\end{frame}

\begin{frame}[t]
\frametitle{Technologies to use}
\begin{center}
\includegraphics[height=6cm]{architecture}
\end{center}
\end{frame}

\begin{frame}[t]
\frametitle{Ubuntu Software Center}
\begin{center}
\includegraphics[height=6cm]{sw_centre}

\small Ubuntu Software Centre screenshot from Wikipedia
\end{center}
\end{frame}

\begin{frame}[t]
\frametitle{Ubuntu Software Center in 2011}
\begin{itemize}
\item looked like what users want
\item closely tied to .deb
\item CLA needed for contribution and not going away
\item bazaar
\item Gtk
\item Python
\end{itemize}
\end{frame}

\begin{frame}[t]
\frametitle{Ubuntu Software Center changes}
\begin{itemize}
\item GSoC 2011 - packagekit backend
\begin{itemize}
\item contributed by student back to Ubuntu
\end{itemize}
\item Boosters sprint in 2012
\begin{itemize}
\item forked and migrated to gitorious Appstream repo
\item polished packagit backend
\item created openSUSE configuration
\item packaged it for openSUSE
\item[\(\Rightarrow\)] almost works in openSUSE
\end{itemize}
\item GSoC 2012 - ???
\end{itemize}
\end{frame}

\begin{frame}[t]
\frametitle{Apper}
\begin{center}
\includegraphics[height=6cm]{apper}

\small Apper screenshot from kde-apps
\end{center}
\end{frame}

\begin{frame}[t]
\frametitle{Apper}
\begin{itemize}
\item Qt package mannager on top of packagekit
\item some efforts started to make use of extra metadata
\item attempts to make it Qt alternative to software cennter
\item upstream works on different approach
\end{itemize}
\end{frame}

\begin{frame}[t]
\frametitle{Distromatch}
\begin{itemize}
\item written by Enrico Zini from Debian
\item matches package names between distributions
\item has API for queries
\item needed for:
\begin{itemize}
\item find screenshots from other distribution
\item find reviews from other distributions
\item find patches from other distributions
\end{itemize}
\end{itemize}
\end{frame}

\begin{frame}[t]
\frametitle{Distromatch - how does it work}
Look into packages for
\begin{itemize}
\item package name
\item desktop files
\item binaries ([/usr]/bin/*)
\item pkg-config files
\item shared libraries
\item man pages
\item devel files
\item stemmed versions of above
\end{itemize}
\end{frame}

\begin{frame}[t]
\frametitle{What is in openSUSE?}
\begin{itemize}
\item obs generates metadata
\item Software Center dependencies are in
\item Software Center just in obs, needs polishing
\item new package search uses metadata and screenshots
\end{itemize}
\end{frame}

\sectionpage{Thank you! Questions?}

\end{document}

