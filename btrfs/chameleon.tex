\documentclass{beamer}
\usetheme[pageofpages=of,% String used between the current page and the
                         % total page count.
          bullet=circle,% Use circles instead of squares for bullets.
          titleline=true,% Show a line below the frame title.
	  titlepagelogo=opensuse,
          alternativetitlepage=true,% Use the fancy title page.
          ]{Torino}

\setbeamerfont{title}{series=\bfseries,size=\LARGE}
\author{Michal Hru\v{s}eck\'{y}\newline {\small openSUSE Boosters}}
\title{Btrfs and Snapper}

\begin{document}

\begin{frame}[t,plain]
\titlepage
\end{frame}

\sectionpage{Btrfs}

\begin{frame}
\frametitle{What is Btrfs?}
\begin{center}
\Huge New super cool filesystem!!!
\end{center}
\end{frame}

\begin{frame}[t]
\frametitle{What's so cool?}
\begin{itemize}
   \item supports several subvolumes per filesystem
   \item supports filesystem over multiple block devices
   \item transparent compression available
   \item filesystem snapshots
   \item cow not only for snapshot but for files as well
   \item online block device addition and removal
   \item online volume growth/shrink
   \item checksums for data and metadata
\end{itemize}
\end{frame}

\begin{frame}[t]
\frametitle{What else?}
\begin{center}
\begin{itemize}
   \item in place ext3/4 conversion
   \begin{itemize}
      \item convert your old partitions!
      \item became just on of the snapshots
      \item returning to ext3/4 possible
   \end{itemize}
   \item no check/recovery tool yet
   \begin{itemize}
      \item you can lost your data 
   \end{itemize}
\end{itemize}
\end{center}
\end{frame}

\sectionpage{Snapper}

\begin{frame}[t]
\frametitle{What does Snapper do?}
\begin{itemize}
   \item comes with command line and YaST interface
   \item tool for managing Btrfs snapshots
   \begin{itemize}
      \item can create snapshots
      \item can delete snapshots
      \item can show difference between snapshots
      \item can mount snapshots
      \item can list snapshots including type and description
      \item comes with zypper an YaST integration
   \end{itemize}
\end{itemize}
\end{frame}

\begin{frame}[t]
\frametitle{Types of snapshot}
\begin{itemize}
   \item \textbf{single}
   \begin{itemize}
      \item simple snapshots with no realation to the others
   \end{itemize}
   \item \textbf{timeline}
   \begin{itemize}
      \item created automatically by cron
   \end{itemize}
   \item \textbf{pre/post}
   \begin{itemize}
      \item created automatically by YaST or zypper
      \item pre snapshot is created before action
      \item post snapshot is created after action
      \item post snapshots knows, who's it's pre snapshot
   \end{itemize}
\end{itemize}
\end{frame}

\begin{frame}[t]
\frametitle{Types of clean ups}
\begin{itemize}
   \item \textbf{number}
   \begin{itemize}
      \item keeps only \textit{number} of newest snapshots
      \item option \texttt{NUMBER\_CLEANUP} in sysconfig file
   \end{itemize}
   \item \textbf{timeline}
   \begin{itemize}
      \item almost same as previous
      \item keeps also \textit{number} of every type of timeline snapshot
      \item option \texttt{TIMELINE\_CLEANUP} in sysconfig file
   \end{itemize}
   \item \textbf{empty-pre-post}
   \begin{itemize}
      \item deletes pre/post pairs when there is no change
      \item option \texttt{EMPTY\_PRE\_POST\_CLEANUP} in sysconfig file
   \end{itemize}
\end{itemize}
\end{frame}

\end{document}

